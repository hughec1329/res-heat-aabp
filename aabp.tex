\documentclass[12pt]{article}
\usepackage{graphicx}
\title{AABP foundation Competitive Research Proposal}
\author{Hugh Crockford}
\date{\today}
\linespread{1.5}
\begin{document}
	\maketitle
	\tableofcontents
	
	\newpage
	\begin{abstract}
		To estimate the effects of calf hutch microenvironment on calf health outcomes.
	\end{abstract}
	\section{Goal and Objectives}
	To utilize a dataset of environmental data (temperature and humidity logged every 5 minutes) collected over 12 months at a commercial calf rearing facility to assess the effect of calf hutch microenvironment on calf health outcomes.
	The Objectives of this project are as follows:
	\begin{itemize}
		\item To investigate the actual microenvironments calves are exposed over the course of a year in a typical managment setting in the San Juaquin Valley.
		\item To compare hutch environments predicted from local weather monitoring stations to actual hutch microenvironment.
		\item To investigate what effect hutch microenvironemnt has on calf health outcomes, as measured by treatment.
	\end{itemize}


	\newpage
	\section{Justification}
	Heat stress is one of the most important diseases affecting modern animal agriculture, causing US livestock industries an estimated \$2.4 billion annual loss, of which the dairy industry accounts for the largest portion, at an estimated loss of \$897 million annually\cite{St-Pierre2003}. 
	These lossess are mostly due to reduced milk yield, impaired reproduction, and increased susceptibility to infectious diseases and consequently increased mortalities.\cite{Kadzere2002,Hammami2013}\\
	Heat stress is likely to become more important in the future due to the impacts of climate change, which is predicted to increase the occurance of extreme weather and rapid temperature changes\cite{Parry2007}. A study utilising various climate models predicted a 10\% milk yield drop and a substantial 35\% drop in conception rate due to the effects of climate change, with the greatest effects felt in the important dairying areas of south west USA\cite{Klinedinst1993}.


	Most of this previous research focuses on the effects of heat stress in Mature cows, who are housed and milked in areas with a variety of environmental modifications ( fans, sprinklers,shade etc.)  to alleviate the effects of heat stress.\cite{Armstrong1994}
	Little thought or research effort has been devoted to assessing the effect of heat stress on the future milking herd, the calves, even though heat stress has shown to have various detrimental effects on calf rearing. \cite{Stott1976,lacetera1994}\\


	The proposed study utilises electronic temperature and humidity devices to log calf level hutch microenvironments over a 12 month period on a commercial dairy. This data will be used to quantify actual conditions experienced by calves, compare recorded conditions to those measured with both local and county wide weather stations, and compare environmental conditions to calf health records to estimate the effect of environment on calf health. 

	\newpage
	\section{Background information, pertinent literature review}
	Hutch housing is the most common housing envirnoment of calves in the United States, with nearly 75\% of dairy operations utilising some form of individual pen/hutch.\cite{NAHMS2007}
	This is for good reason, as the Individual calf hutch placed outdoors is the best environment for prevention of respiratory disease \cite{callan2002biosecurity}, and and calves raised in hutches are less likely to be treated for scours. \cite{Waltner-Toews1986}
	Hutches have the advantage that calves can move between three distinct microenvironments in the rear of hutch, front of hutch, and outdoor area ~\cite{brunsvold1985} \\
	This advantage is soon lost as the temperature rises, and the calf becomes hyperthermic. Calves have a quite narrow thermoneutral zone of 10 - 20 degrees celcius \cite{Gebremedhin1981}, and besides having a much narrower range that mature cows, calves are less able than older animals to regulate their own body temperature.\cite{Christopherson1976}\\


	There has been no research into the microenvironment within the hutch itself, especially under high temperatures. Temperature heat indexes are often used to assess and predict the impact of heat stress on mature cow production, 
	Hot hutch environments have been shown to negatively impact Serum IgG absorbtion \cite{Stott1976}. Almost 20\% of dairy calves have failure of passive transfer (<10mg/mL measured by RID at 1-7 days old)\cite{NAHMS2007}, so any further impairment of IgG absorbtion is very likely to impact future health outomces.\cite{Besser1994}\\
	A hot environment may negatively impact early growth of calves, leading to reduced production as adults \cite{Hoffman1997}. A study by Lacetera found lower wither height, hip width, and Body Condition Score in 5 month old Holstein calves that were exposed to hot conditions early in life \cite{lacetera1994} \\
	There has been some investigation into the effects of hutch microenvironments on calf health, \cite{Stott1976,Nordlund2008,Lago2006}, however these studies differ  in many important ways from that proposed, which make its application to the California Dairy industry difficult. 
	Stott's study was conducted in Arizona, which has a similar hot dry climate during the summer, However the study calves were housed in corrugated iron hutches in full sun, with solid sidewalls which would severely restrict ventilation\cite{Stott1976}. This is in comparison to the typical wooden 'California style' triplet calf hutch widely used on dairies in the San Juaquin valley. The Wooden side walls will reduce the amount of radiant heat transferred to the calves, and the common design featuring missing panels on the side and back would dramatically improve ventilation, an important factor in both reducing temperature and disease control\cite{Smith2002a}.\\
	Other studies have focused on calves housed in calf barns, a common housing method in the midwest where winter temperatures make hutch housing impractical \cite{Lago2006}. The environment within these calf barns is very different to the calf hutches in califonia, and obviously the environmental differences make it difficult to draw direct comparisons from these studies.


	The proposed study alleviates these issues due to data collection taking place on a commercial dairy, and hence findings are directly applicable to the hundreds of thousands of calves raised in a similar manner in california every year. %millions?

	\newpage
	\section{Materials and Methods}
		The dataset to be used in this project was previously collected by . . .  %CITATION?, 
		and features collected over a 12 month period (October 2003 to October 2004). While the environmental data was being colleced, ~25,000 calf treatment records were also obtained for the same period, featuring calf id's and treatment given. 
		During data collection, a commercial weather station was also placed on the calf rearing facility. The information collected by this station will be compared to the data gathered from each calf hutch, to analyse . . . . . . . The data collected from this station will also be compared to data available from government weather stations at locations surrounding the calf rearing facility, to determine if local on farm conditions can be predicted from a combination of nearby government weather stations. These findings could be used to predict farm-level heat stress levels and forecasts, allowing improved managment of both calves and mature cows.
		These results will be combined to form a heat stress monitoring and prediction software to be hosted on a server at the VMTRC, UC Davis. This data can be accessed by local stakeholders via a webiste, automatic email updates/alerts, and smartphone applications.
		The number of treatments given, as a proxy for calf health would form the respose variable of our predictor model.

		Hobo data loggers have been used extensively in similar projects and proved a reliable measure of actual environmental conditions\cite{Scharf2011,Jousan2007}.
	\newpage
	\bibliographystyle{plain}
	\bibliography{/home/hugh/library}
\end{document}
