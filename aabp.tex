\documentclass[12pt]{article}
\usepackage{graphicx}
\title{AABP foundation Competitive Research Proposal}
\author{Hugh Crockford}
\date{\today}
\linespread{1.33}
\begin{document}
	\maketitle
	\tableofcontents
	
	\newpage
	\begin{abstract}
		To estimate the effects of calf hutch microenvironment on calf health outcomes.
	\end{abstract} 
	\section{Goal and Objectives}
	The goal of this project is to utilize a dataset of environmental data collected over 12 months at a commercial calf rearing facility to assess the effect of calf hutch microenvironment on calf health outcomes.\\


	The Objectives of this project are as follows:
	\begin{enumerate}
		\item To investigate the actual microenvironments calves are exposed to over the course of a year in a typical management setting in the San Joaquin Valley.
		\item To compare hutch environments predicted from local and county wide NOAA weather monitoring stations to actual hutch microenvironment.
		\item To investigate what effect hutch microenvironment has on calf health outcomes, as measured by treatment. % to quantify?
		\item To combine research findings into an accessible form that can be utilised by dairy producers in the day to day running of their operations.
	\end{enumerate}


	\newpage
	\section{Justification}
	Heat stress is one of the most important diseases affecting modern animal agriculture, causing US livestock industries an estimated \$2.4 billion annual loss, of which the dairy industry accounts for the largest portion, at an estimated loss of \$897 million annually\cite{St-Pierre2003}. 
	These losses are mostly due to reduced milk yield, impaired reproduction, and increased susceptibility to infectious diseases and consequently increased mortalities.\cite{Kadzere2002,Hammami2013}\\
	Heat stress is likely to become more important in the future due to the impacts of climate change, which is predicted to increase the occurrence of extreme weather and rapid temperature changes\cite{Parry2007}. A study utilising various climate models predicted a 10\% milk yield drop and a substantial 35\% drop in conception rate due to the effects of climate change, with the greatest effects felt in the important dairying areas of south west USA\cite{Klinedinst1993}.\\


	Most of the previous research has focused on the effects of heat stress in mature cows, who are housed and milked in areas with a variety of environmental modifications (fans, sprinklers, shade etc.)  to alleviate the effects of heat stress.\cite{Armstrong1994}
	Little research effort has been devoted to assessing the effect of heat stress on the future milking herd, the calves, even though heat stress has shown to have various detrimental effects on calf rearing, and consequently future production. \cite{Stott1976,lacetera1994,Hoffman1997}\\


	The proposed study utilises electronic temperature and humidity recording devices to log calf level hutch microenvironments over a 12 month period on a commercial dairy. This data will be used to quantify actual conditions experienced by calves, compare recorded conditions to those measured with both local and county wide weather stations, and compare environmental conditions to calf health records to estimate the effect of environment on calf health. 

	\newpage
	\section{Background information, pertinent literature review}
	Hutch housing is the most common housing environment of calves in the United States, with nearly 75\% of dairy operations utilising some form of individual pen/hutch.\cite{NAHMS2007}
	This is for good reason, as the individual calf hutch placed outdoors is the best environment for prevention of respiratory disease \cite{callan2002biosecurity}, and calves raised in hutches are less likely to be treated for scours. \cite{Waltner-Toews1986}
	Hutches have the advantage that calves can move between three distinct microenvironments in the rear of hutch, front of hutch, and outdoor area ~\cite{brunsvold1985} 
	This advantage is soon lost as the temperature rises, and the calf becomes hyperthermic. Calves have a quite narrow thermoneutral zone of 10 - 20 degrees Celsius \cite{Gebremedhin1981}, and besides having a much narrower range that mature cows, calves are less able than older animals to regulate their own body temperature.\cite{Christopherson1976}\\


	Environment is very important to calf rearing as it is the most readily manipulated part of the epidemiological triad (consisting of Host, Pathogen, and Environment\cite{CDC2012}), and any suboptimal environmental effects will increase the likelihood of calves developing disease. 
	Environment has shown to be particularly important in the pathogenesis of Bovine Respiratory disease, \cite{Lago2006} which is responsible for almost half of the preweaning deaths in dairy heifers \cite{NAHMS2007}. 
	Quantifying the conditions calves are actually exposed to is the first step to managing the environmental stressors that places these animals at increased risk of developing respiratory disease. 
	

	There has been little research into the microenvironment actually experienced by the calves within the hutch, especially under high temperatures. 
	Within mature cows, temperature heat indexes are often used to assess and predict the impact of heat stress on production\cite{Bohmanova2007}, but these indexes apply to an adult animal in a very different environment from a hutch, so are of little value in calves.
	It would be useful to have a similar heat index that predicts when calves are under environmental stress that could lead to increased disease and reduced growth.


	Hot hutch environments have been shown to negatively impact Serum IgG absorption \cite{Stott1976}. Almost 20\% of dairy calves have failure of passive transfer (\< 10mg/mL measured by RID at 1-7 days old)\cite{NAHMS2007}, so any further impairment of IgG absorption is very likely to impact future health outcomes.\cite{Besser1994}\\
	A hot environment may also negatively impact early growth of calves, leading to reduced production as adults \cite{Hoffman1997}. A study by  Lacetera found lower wither height, hip width, and Body Condition Score in 5 month old Holstein calves that were exposed to hot conditions early in life \cite{lacetera1994} \\
	\newpage
	There has been some investigation into the effects of hutch microenvironments on calf health, \cite{Stott1976,Nordlund2008,Lago2006}, however these studies differ  in many important ways from that proposed, which make their application to the California Dairy industry difficult.\\ 
	Stott's study was conducted in Arizona, which has a similar climate to California, however the study calves were housed in corrugated iron hutches in full sun, with solid sidewalls which would severely restrict ventilation\cite{Stott1976}. This is in comparison to the typical wooden 'California style' triplet calf hutch widely used on dairies in the San Joaquin valley. The wooden side walls will reduce the amount of radiant heat transferred to the calves, and a common design feature of missing panels on the side and back would dramatically improve ventilation, an important factor in both reducing temperature and disease control\cite{Smith2002a}.\\
	Other studies have focused on calf barns, a common housing method in the Midwest where winter temperatures make hutch housing impractical \cite{Lago2006}. The environment within these calf barns is very different to the calf hutches in California, which makes it difficult to draw direct comparisons from these studies.\\


	The proposed study  utilises data collected on a commercial dairy in the California central valley, using calves in the most common form of housing, and hence findings are directly applicable to the millions of calves raised in a similar manner in California every year. %millions?

	\newpage
	\section{Materials and Methods}
		The dataset to be used in this project was previously collected by . . .  %CITATION?, 
		and features temperature and humidity measurements every 30 seconds over a 12 month period (October 2003 to October 2004). 
		Hobo data loggers\cite{Onset2012} were used to collect these measurements, and have been used extensively in similar projects and proved a reliable measure of actual environmental conditions\cite{Scharf2011,Jousan2007}.


		While the environmental data was being collected, ~25,000 calf treatment records were also obtained for the same period, featuring calf id's and treatment given. 
		The number of treatments given, as a proxy for calf health would form the response variable of our predictor model. % model description?


		Local weather conditions were also monitored during the data collection period by a commercial weather station which was placed on the calf rearing facility. The information collected by this station will be compared to the data gathered from within the calf hutches, to analyse the accuracy of predicting hutch microenvironment from weather station data. 
		The on farm weather station records will also be compared to data available from government weather stations at locations surrounding the calf rearing facility\cite{NOAA2012}, to determine if local on farm conditions can be predicted from a combination of nearby government weather stations. 


		These findings could be used to predict farm-level heat stress levels and forecasts, allowing improved management of both calves and mature cows.\\
		These results will be combined combined with the results of another project %cite the grant?
		to form a heat stress monitoring and prediction application to be hosted on a server at the VMTRC, UC Davis. 
		This data can be freely accessed by local stakeholders via a website, automatic email updates/alerts, and smartphone applications that utilise GPS facilities to give an accurate localised heat stress forecast.


	\newpage
	\bibliographystyle{unsrt}
	\bibliography{/home/hugh/Documents/library}
\end{document}
