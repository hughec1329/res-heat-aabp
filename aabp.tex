\documentclass[12pt]{article}
\usepackage{graphicx}
\title{AABP foundation Competitive Research Proposal}
\author{Hugh Crockford}
\date{\today}

% aims:
%% 1. does temp/humid differ signifigantly accross hutch farm?
%% 2. is the diff env's associated with diff health outcomes.


\begin{document}
	\begin{abstract}
	
	\end{abstract}
	\section{Goal and Objectives}
	\section{Justification}
	Hutch housing is the most common housing envirnoment of calves in the United States, with nearly 75\% of dairy operations utilising some form of individual pen/hutch.\cite{NAHMS2007}
	The Individually calf hutched placed outdoors is the best environment for prevention of respiratory disease \cite{callan2002biosecurity}, and and Calves raised in hutches are less likely to be treated for scours. \cite{Waltner-Toews1986}
	Hutches have the advantage that calves can move between three distinct microenvironments in the rear of hutch, front of hutch, and outdoor area ~\cite{brunsvold1985} \\
	This advantage is soon lost as the temperature rises, and the calf becomes hyperthermic. Calves have a quite narrow thermoneutral zone of 10 - 20 degrees celcius \cite{Gebremedhin1981}
	This is especially in young animals, as they are less able than older animals to regulate their own body temperature.\cite{Christopherson1976}
	% Heat stress has long been recognised as reducing production\cite, reproduction\cite{ } , and growth rates of cattle.
	There has been no research into the microenvironment within the hutch itself, especially under high temperatures.
	Hot hutch environments have been shown to negatively impact Serum IgG absorbtion \cite{Stott1976} . This 
	A hot environment may negatively impact early grow the of calves. 
	A study by Lacetera found lower wither height, hip width, and Body Condition Score in 5 month old Holstein calves that were exposed to hot conditions early in life \cite{lacetera1994} \\
	This study, While similar to that proposed, has many shortcomigs which make its application to the California Dairy industry difficult. 
	The study was conducted in Arizona, which has a similar hot dry climate during the summer.
	The study calves were housed in corrugated iron hutches in full sun, with solid sidewalls which would severely restrict ventilation. This is in comparison to the typical wooden 'California style' triplet calf hutch widely used on dairies in the San Juaquin valley and beyond. The Wooden side walls will reduce the amount of radiant heat transferred to the calves, and the frequent addition of missing panels in side and back would dramatically improve ventilation, an important factor in both reducing temperature and disease control % \cite need citation of ventilation = good.
	There have been no other studies to compare hutch microenvirnoment to calf health and growth outcomes.

	\section{Background information, pertinent literature review}
	\section{Materials and Methods}
	\bibliographystyle{plain}
	\bibliography{/home/hugh/library}
\end{document}
